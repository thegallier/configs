\documentclass[11pt, a4paper]{article}

% --- Packages ---
\usepackage[utf8]{inputenc} % Input encoding
\usepackage[T1]{fontenc}    % Font encoding
\usepackage{amsmath}        % AMS math package
\usepackage{amssymb}        % AMS symbols
\usepackage{amsfonts}       % AMS fonts
\usepackage[margin=1in]{geometry} % Set margins
\usepackage{hyperref}       % Clickable links (optional)
\usepackage{enumitem}       % For list customization (optional)

% --- Document Information ---
\title{Summary: Convex Polynomial Price Adjustment Optimization}
\author{Generated by AI Assistant} % Optional
\date{\today}                 % Optional

% --- Begin Document ---
\begin{document}

\maketitle

\section*{Objective}
To develop and analyze a price adjustment model for trades based on customer tier and trade DV01. The model utilizes convex and monotonic Bernstein polynomials, allowing for flexible yet controlled adjustments. The goal is to optimize the scaling of these polynomial adjustments to maximize the Profit \& Loss (P\&L) generated from \textit{winning} trades, subject to a user-defined constraint on the \textit{Losing DV01 Ratio}.

\section*{Methodology}
\begin{enumerate}
    \item \textbf{Input Data:}
        \begin{itemize}[nosep] % nosep reduces space between items
            \item Trade data including \texttt{customerName}, \texttt{tier}, \texttt{firmAccount}, \texttt{cusip}, \texttt{amount}, \texttt{mid} price, execution \texttt{side} (BUY/SELL), \texttt{tradePrice}, and \texttt{dv01}.
            \item A per-\texttt{cusip} sensitivity factor $\epsilon$ (fixed at 0.1 in this implementation).
        \end{itemize}

    \item \textbf{Polynomial Modeling:}
        \begin{itemize}[nosep]
            \item Two 1-dimensional Bernstein polynomials are used as the basis for adjustments:
                \begin{itemize}
                    \item $f_1(t)$: Based on normalized customer \texttt{tier} ($tier_{\text{norm}} \in [0, 1]$).
                    \item $f_2(d)$: Based on normalized \texttt{dv01} ($dv01_{\text{norm}} \in [0, 1]$).
                \end{itemize}
            \item The polynomials are defined by their control points (coefficients) $C^{(1)}_k$ and $C^{(2)}_k$, respectively.
            \item Two rescaling factors, $r_1$ and $r_2$, are introduced to globally scale the output of the base polynomials.
        \end{itemize}

    \item \textbf{Polynomial Fitting (Base Coefficients):}
        \begin{itemize}[nosep]
            \item The user provides target points (4 initial Z-values for each polynomial) via sliders. These define target shapes.
            \item The base polynomial coefficients ($C^{(1)}_k$, $C^{(2)}_k$) are determined by solving separate optimization problems for each polynomial:
                \begin{itemize}
                    \item \textbf{Objective:} Minimize the sum of squared errors between the polynomial output and the user-defined target Z-values at the corresponding normalized X-coordinates.
                    \item \textbf{Constraints:}
                        \begin{itemize}
                            \item \textit{Convexity:} Enforced for both polynomials ($\Delta^2 C \ge 0$). Ensures the second derivative is non-negative.
                            \item \textit{Monotonicity (Increasing):} Enforced \textit{only} for the DV01 polynomial ($f_2$) ($\Delta C \ge 0$). Ensures a larger DV01 results in a larger (or equal) multiplier basis.
                        \end{itemize}
                    \item \textbf{Solver:} CVXPY with a suitable solver (e.g., SCS).
                \end{itemize}
        \end{itemize}

    \item \textbf{Price Adjustment Simulation:}
        \begin{itemize}[nosep]
            \item For each trade $i$:
                \begin{itemize}
                    \item Normalize $tier_i$ and $dv01_i$.
                    \item Calculate base polynomial outputs: $v_{1,\text{base}} = f_1(tier_{\text{norm},i})$, $v_{2,\text{base}} = f_2(dv01_{\text{norm},i})$.
                    \item Apply rescaling: $value_1 = r_1 \cdot v_{1,\text{base}}$, $value_2 = r_2 \cdot v_{2,\text{base}}$.
                    \item Calculate adjustment: $adjustment_i = \epsilon_i \cdot value_1 \cdot value_2 \cdot \text{sign}(\text{side}_i)$.
                    \item Calculate adjusted price: $price_{\text{adj},i} = mid_i + adjustment_i$.
                \end{itemize}
        \end{itemize}

    \item \textbf{Metrics Calculation:}
        \begin{itemize}[nosep]
            \item \textit{Win Condition:} Trade $i$ is 'winning' if $price_{\text{adj},i} > price_{\text{trade},i}$ (BUYs), or $price_{\text{adj},i} < price_{\text{trade},i}$ (SELLs).
            \item \textit{Losing DV01 Ratio:} $\frac{\sum_{i \in \text{Losing}} dv01_i}{\sum_{\text{all } i} dv01_i}$.
            \item \textit{Potential P\&L (per trade vs mid):} $pnl_{\text{pot},i} = amount_i \cdot (mid_i - price_{\text{adj},i}) \cdot \text{sign}(\text{pnl}_i)$.
            \item \textit{Actual Winning P\&L:} $\sum_{i \in \text{Winning}} pnl_{\text{pot},i}$.
            \item \textit{Potential P\&L (All Trades):} $\sum_{\text{all } i} pnl_{\text{pot},i}$.
            \item \textit{P\&L (Favorable Adjustment):} $\sum_{i \text{ where Adj Favors}} pnl_{\text{pot},i}$, where 'Adj Favors' means $price_{\text{adj}} \le mid$ (BUYs) or $price_{\text{adj}} \ge mid$ (SELLs).
            \item \textit{Efficiency Ratio:} $\frac{\text{Actual Winning P\&L}}{\text{Potential P\&L (All Trades)}}$.
        \end{itemize}

    \item \textbf{Optimization:}
        \begin{itemize}[nosep]
            \item \textbf{Goal:} Find optimal rescaling factors $r_1^*, r_2^*$.
            \item \textbf{Objective:} Maximize Actual Winning P\&L.
            \item \textbf{Constraint:} $| \text{Losing DV01 Ratio} - \text{Target Ratio} | \le \text{Tolerance}$. The Target Ratio is user-defined.
            \item \textbf{Bounds:} $0.5 \le r_1, r_2 \le 2.0$.
            \item \textbf{Solver:} \texttt{scipy.optimize.minimize} using the \texttt{COBYLA} method.
        \end{itemize}

    \item \textbf{Visualization (Dashboard):}
        \begin{itemize}[nosep]
            \item Interactive sliders for initial polynomial Z-values and rescaling factors ($r_1$, $r_2$).
            \item Input fields for polynomial \texttt{degree} and optimization Target Ratio.
            \item Plots showing initial/current PWL points, base fitted polynomial, and rescaled polynomial (on original Tier/DV01 axes).
            \item Display of global metrics (Losing DV01 Ratio, various P\&Ls, Efficiency).
            \item Heatmaps showing Losing DV01 Ratio and Efficiency Ratio across a grid of $r_1, r_2$ values.
            \item Comparison tables showing initial vs. current metrics aggregated by CUSIP, Tier, and Customer, with sortable delta columns.
            \item Download button for the results DataFrame.
        \end{itemize}
\end{enumerate}

\section*{Key Formulas}
\begin{itemize}
    \item \textbf{Bernstein Basis Polynomial:}
    \begin{equation*}
        B_{k,n}(t) = \binom{n}{k} t^k (1-t)^{n-k}, \quad t \in [0, 1]
    \end{equation*}
    \item \textbf{Polynomial Evaluation (Base):}
    \begin{equation*}
        f(t_{\text{norm}}) = \sum_{k=0}^{n} C_k B_{k,n}(t_{\text{norm}})
    \end{equation*}
    \item \textbf{Rescaled Values:}
    \begin{align*} % Use align* for multi-line unnumbered equations
        value_1 &= r_1 \sum_{k=0}^{n_1} C^{(1)}_k B_{k,n_1}(tier_{\text{norm}}) \\
        value_2 &= r_2 \sum_{k=0}^{n_2} C^{(2)}_k B_{k,n_2}(dv01_{\text{norm}})
    \end{align*}
    \item \textbf{Adjustment:}
    \begin{equation*}
        adjustment = \epsilon \cdot value_1 \cdot value_2 \cdot \text{sign}(\text{side})
        \quad \text{where sign}(\text{BUY}) = -1, \text{sign}(\text{SELL}) = +1
    \end{equation*}
    \item \textbf{Adjusted Price:}
    \begin{equation*}
        price_{\text{adj}} = mid + adjustment
    \end{equation*}
    \item \textbf{Convexity Constraint:}
    \begin{equation*}
        C_{k+2} - 2C_{k+1} + C_k \ge 0 \quad \forall k \in \{0, ..., n-2\}
    \end{equation*}
    \item \textbf{Monotonicity Constraint (Increasing):}
    \begin{equation*}
        C_{k+1} - C_k \ge 0 \quad \forall k \in \{0, ..., n-1\}
    \end{equation*}
    \item \textbf{Optimization Problem:}
    \begin{equation*}
        \max_{0.5 \le r_1, r_2 \le 2.0} \left( \sum_{i \in \text{Winning}(r_1, r_2)} pnl_{\text{pot},i}(r_1, r_2) \right)
    \end{equation*}
    subject to:
    \begin{equation*}
        \left| \frac{\sum_{i \in \text{Losing}(r_1, r_2)} dv01_i}{\sum_{\text{all } i} dv01_i} - \text{Target Ratio} \right| \le \text{Tolerance}
    \end{equation*}
\end{itemize}

\section*{Conclusion}
This approach provides a structured way to model price adjustments using constrained polynomials derived from simple user inputs. It allows for analysis of trade outcomes under different scaling assumptions and enables optimization to find parameters that best meet a defined objective (maximizing winning P\&L) under a quantifiable risk constraint (Losing DV01 Ratio). The interactive dashboard facilitates exploration and understanding of the model's behavior.

\end{document}
